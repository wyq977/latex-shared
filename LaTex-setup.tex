\documentclass[aos,preprint]{imsart}

\input{macro}

\usepackage[colorlinks,citecolor=blue,urlcolor=blue,colorlinks=true,linkcolor=blue]{hyperref}
\usepackage{caption}                             % show plain tex
\usepackage{fancyvrb}                            % extended verbatim environments
\fvset{fontsize=\footnotesize}                   % default font size for fancy-verbatim en
% https://stackoverflow.com/questions/55036409/latex-verbatimhow-to-show-the-file-directoy-created-by-tree-command
\usepackage[utf8]{inputenc}
\usepackage{pmboxdraw}

\begin{document}

\title{\LaTeX\ Project Management}

To be consistent for most \LaTeX\ projects,
I choose AOS's preprint version as the default, please check
\url{https://vtex-soft.github.io/texsupport.ims-aos/} for more details.

The references are managed externally by Zotero and BBT,
exported to BibTex format, then included via \verb|natbib|.

\section{Font size warnings}

Sometimes you will see warnings like
\begin{Verbatim}
LaTeX Font Warning: Font shape `OML/cmm/m/it' in size <5.5> not available
(Font)              size <5> substituted on input line 50.
\end{Verbatim}

It is usually solved by \verb|lmodern|, \verb|anyfontsize| and \verb|fix-cm| (which I used) packages.

\section{Formatting}

Ended up using the \verb|latexindent| that comes with MacTeX distribution.
It can be updated in TeX Live Utility.app, the default settings can be found in \verb|indent.log|:
\begin{Verbatim}
/usr/local/texlive/2025/texmf-dist/scripts/latexindent/defaultSettings.yaml
\end{Verbatim}

To modify, I created a file \verb|~/.indentconfig.yaml| as:
\begin{Verbatim}
# Paths to user settings for latexindent.pl
paths:
- ~/projects/latex-shared/latexindent.yaml
\end{Verbatim}

The file \verb|latexindent.yaml| is managed in my git repo.
Other settings: \url{https://latexindentpl.readthedocs.io/en/latest/index.html}.

\section{Project structure}

\subsection{Shared files}

Currently, I have my files used across different \LaTeX\ saved in one
github repo where I make symbolic link so that I could use shortcuts like probability
$\PR{X}$ directly by including directly.

\subsection{Single file}

This is the default with one-file project, easy to track.

\begin{Verbatim}
	├── fig
	│   ├── plots.pdf
	├── main.bib
	├── main.tex
	├── main.pdf
	├── marco.tex             % All my collected macros
	├── custom-style.cls/def/sty/bst
\end{Verbatim}

\subsection{Multi-files}

Below is a larger \LaTeX\ with different chapters:

\begin{Verbatim}
	├── chapters
	│   ├── 01-blabla.tex
	├── fig
	│   ├── R/Python.pdf
	│   ├── TikZ.tex
	│   ├── TikZ.pdf
	│   ├── Asymptote.asy
	│   ├── Asymptote.pdf
	├── main.bib
	├── main.tex
	├── main.pdf
	├── marco.tex             % All my collected macros
	├── custom-style.cls/def/sty/bst
\end{Verbatim}

% TODO: how to adjust figure size and fonts sizes
% TODO: BibTex or BibLaTex?

\section{Mathematical notation}

It has always been a hassle to organise mathematical notation across different sources,
in fact, I would go so far as to argue that this is the most annoying thing
when one starts reading a book or an article.

However, there \textit{must be} some notational conflicts beyond primary school
simply due to the fact that the limited number of alphabets (\textbf{26}).
For example, ``$\mathbb{E}$'' might be energy in physics
while it could refer to expectation or scores in probability.

Another difficulty is that the authors often assume some familiarity in the
topics
\emph{also} I am expected to read in some logical or chronological order.
In reality, I am constantly jumping back and forth between one literature to
another.

\begin{align*}
	\Af, \Bf, \Cf, \Df, \Ef, \Ff, \Gf, \Hf, \Jf, \Kf, \Lf, \Mf, \Nf, \Of, \Pf,
	\Qf, \Rf, \Sf, \Tf, \Uf, \Vf, \Wf, \Xf, \Yf, \Zf      \\
	\Ac, \Bc, \Cc, \Dc, \Ec, \Fc, \Gc, \Hc, \Ic, \Jc, \Kc, \Lc, \Mc, \Nc, \Oc,
	\Pc, \Qc, \Rc, \Sc, \Tc, \Uc, \Vc, \Wc, \Xc, \Yc, \Zc \\
	\Ak, \Bk, \Ck, \Dk, \Ek, \Fk, \Gk, \Hk, \Ik, \Jk, \Kk, \Lk, \Mk, \Nk, \Ok,
	\Pk, \Qk, \Rk, \Sk, \Tk, \Uk, \Vk, \Wk, \Xk, \Yk, \Zk
\end{align*}
$$
	\arginf, \argsup, \argmax, \argmin, \conv
$$

This stackexchange answer: \url{https://tex.stackexchange.com/a/58124}
is probably the most comprehensive answer to which fonts are shown in \LaTeX.

\begin{tabular}{ccl}
	\textbf{Symbol} & \textbf{Usage} & \textbf{Comments}                                 \\
	$\Bf$           & \verb|\*f|     & blackboard bold except \verb|\If| due to conflict \\
	$\Bc$           & \verb|\*c|     & calligraphic font                                 \\
	$\Bk$           & \verb|\*k|     & Fraktur font                                      \\
\end{tabular}


\subsection{Choose the notations and shortcuts wisely}

I am not even talking about the difference due to differences in fonts and italic or roman.
It is just a very sad thing that we don't even have a unified way of saying probability is just sad.
Take probability for example and return to the most basic case of tossing a
coin
where the sample space is $\Ac = \{H, T\}$. I have come across:

\begin{align*}
	P() \quad \mathrm{P}() \quad \mathrm{Pr}() \quad \mathsf{P}() \quad
	\mathbf{Prob}().
\end{align*}

\end{document}